\documentclass{article}
\usepackage[portuguese]{babel}
\usepackage[utf8]{inputenc}

\title{Resumo do artigo "A Few Useful Things to Know About Machine Learning"}
\author{André Rato[45517], Diogo Faustino[40968], José Alexandre[45223]}
\date{9 de março 2021}

\begin{document}
\maketitle
\begin{abstract}
Neste trabalho faz-se um resumo do artigo intitulado "A Few Useful Things to Know About Machine Learning"\cite{ref1}.

\end{abstract}
\section{Descrição do artigo analisado}
Neste artigo, é feita uma apresentação de alguns conhecimentos importantes adquiridos por investigadores através das suas experiências. Este artigo está dividido em 12 secções.\\
Na primeira secção, são apresentados os 3 componentes que todos os algoritmos de \textsl{machine learning} possuem:
\begin{itemize}
    \item a \textbf{representação} numa \textsl{machine learner} é o conjunto de classificadores/funções que podem ser aprendidos (espaço de hipóteses) - se uma função não está no espaço de hipóteses, não pode ser aprendida;
    \item a função de \textbf{avaliação} avalia o modelo da \textsl{machine learning};
    \item a \textbf{otimização} é o método utilizado para procurar o melhor modelo de aprendizagem.
\end{itemize}
Nas segunda e terceira secções, são focados os temas da generalização dos dados utilizados e da quantidade dos mesmos ser suficiente ou não para a representação e avaliação dos modelos.\\
Overfitting é o tema principal da quarta secção, onde é referido que uma das maneiras de interpretar overfitting é partir o erro de generalização em duas componentes: \textbf{bias} (tendência de a \textsl{machine learner} aprender a mesma coisa errada) e \textbf{variance} (tendência para aprender coisas aleatórias).\\
Na quinta secção, o autor refere que generalizar corretamente torna-se mais difícil consoante a dimensão dos dados. Os algoritmos de \textsl{machine learning} dependem de um raciocínio baseado em similaridade.\\
O autor aborda ainda, na secção 6, que todas as garantias teóricas que possam ser tomadas não são o que aparentam e podem gerar erros no final. Este problema pode ser corrigido recorrendo à informação presente na sétima secção, onde o autor explica que a engenharia de recursos é a chave para o sucesso na área.\\
Como regra geral, um algoritmo pouco inteligente com muitos dados supera um algoritmo muito inteligente com uma quantidade modesta de dados. Porém, quanto mais dados, maior a quantidade de problemas (secção 8).\\
Na nona secção, o autor defende que aprender muitos modelos é melhor do que aprender apenas um.\\
Simplicidade não implica precisão é abordado na secção 10 e a diferença entre representável e aprendível na secção 11.\\
Na última secção do artigo é referido que a relação causa-relação não se aplica à \textsl{machine learning}. 

\begin{thebibliography}{8}
\bibitem{ref1}
Domingos, P. - A Few Useful Things to Know About Machine Learning. Tapping into the “folk knowledge” needed to advance machine learning applications. United States: Andrew A. Chien. 0001-0782. Volume 55 2012, 78--87.
Last accessed 9 March 2021
\end{thebibliography}
\end{document}