\documentclass{article}
\usepackage[utf8]{inputenc}

\title{Robôs Domésticos}
\author{André Rato[45517], Diogo Faustino[40968], José Alexandre[45223]}
\date{31 de março de 2021}

\usepackage[sorting=none]{biblatex}
\bibliography{references}
\PassOptionsToPackage{hyphens}{url}\usepackage{url}

%%% Elimiação do overlow das references %%%
\setcounter{biburllcpenalty}{7000}
\setcounter{biburlucpenalty}{8000}

\begin{document}
\maketitle

\renewcommand\abstractname{Resumo}
\begin{abstract}

O tema que pretendemos abordar é a "Robôs Domésticos", um tema que nos últimos anos tem tido uma enorme evolução, evolução essa que passa despercebida ao nossos olhos.\par
Pretendemos explorar o que são os \textbf{robôs domésticos}\cite{whatare} e conhecer um pouco melhor qual a sua \textbf{história} e \textbf{evolução}\cite{history}. Pretendemos conhecer também quais as \textbf{vantagens e desvantagens}\cite{advantages} da utilização dos mesmos.\par
O motivo para a escolha deste tema é a relação que a robótica tem com a \textbf{inteligência artificial}\cite{airobots} e a \textbf{influência que a robótica tem vindo a ter no mercado financeiro}\cite{market}.\par
Pretendemos também investigar acerca de alguns \textbf{exemplos de robôs domésticos}\cite{examples}, de modo a sabermos as suas funções e ficar a conhecer um pouco sobre o seu funcionamento.

\end{abstract}

\printbibliography[title=Referências]

\end{document}